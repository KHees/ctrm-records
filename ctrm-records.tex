\documentclass[12pt]{article}
\fontfamily{times}
\usepackage{graphicx}
\usepackage{geometry}

\usepackage{amsmath, amssymb, amsthm}
\usepackage{tikz}

\geometry{verbose,tmargin=30mm,bmargin=25mm,lmargin=25mm,rmargin=25mm}
\pagestyle{empty}

\newtheorem{theorem}[equation]{Theorem}
\newtheorem{lemma}[equation]{Lemma}
\newtheorem{proposition}[equation]{Proposition}
\newtheorem{corollary}[equation]{Corollary}
\newtheorem{definition}[equation]{Definition}
\newtheorem{example}[equation]{Example}
\newtheorem{remark}[equation]{Remark}
\newtheorem{question}[equation]{Question}
\newtheorem{notation}[equation]{Notation}
%\numberwithin{equation}{section}

\newcommand{\R}{\mathbb{R}}
\newcommand{\Q}{\mathbb{Q}}
\newcommand{\C}{\mathbb{C}}
\newcommand{\N}{\mathbb{N}}
\newcommand{\F}{\mathbb{F}}
\newcommand{\PP}{\mathbb{P}}
\newcommand{\T}{\mathbb{T}}
\newcommand{\Z}{\mathbb{Z}}
\newcommand{\B}{\mathfrak{B}}
\newcommand{\BB}{\mathcal{B}}
\newcommand{\M}{\mathfrak{M}}
\newcommand{\X}{\mathfrak{X}}
\newcommand{\Y}{\mathfrak{Y}}
\newcommand{\CC}{\mathcal{C}}
\newcommand{\E}{\mathbb{E}}
\newcommand{\cP}{\mathcal{P}}
\newcommand{\cS}{\mathcal{S}}
\newcommand{\A}{\mathcal{A}}
\newcommand{\ZZ}{\mathcal{Z}}

%Peter's commands
\newcommand{\ex}{\mathbb {E}}
\newcommand{\pr}{\mathbb {P}}
\newcommand{\Rd}{\mathbb R^d}
\newcommand{\Rp}{\mathbb R^+}
\newcommand{\spctim}{\mathbb R^{d+1}}
\newcommand{\del}{\partial }
\newcommand{\1}{\mathbf 1}
\newcommand{\eps}{\varepsilon}

%Other commands
\newcommand{\FF}{\mathcal{F}}
\newcommand{\Law}{\mathop{\rm Law}}
\newcommand{\Cov}{\mathop{\rm Cov}}
\newcommand{\Var}{\mathop{\rm Var}}
\newcommand{\sign}{\mathop{\rm sign}}
\newcommand{\Floor}[1]{{\lfloor {#1} \rfloor}}
\newcommand{\cd}{\overset{d}{\longrightarrow}}
\newcommand{\D}{\mathbb{D}}
\newcommand{\ppartial}[2]{\dfrac{\partial {#1}}{\partial {#2} }}
\newcommand{\cdj}{\overset{d}{\underset{J_1}{\longrightarrow}}}

\title{Exceedances and Records Heavy-Tailed Max-Renewal Processes}
\author{Nathan Giang \and Katharina Hees \and Peter Straka}



\begin{document}

\maketitle

\begin{abstract}
Extreme Value theory deals with the observation of extreme events which are the maxima of a sequence of observations and admits that these events occur at regular intervals in time. Recently a new theory called Continuous Time Random Maxima gets much regard which can be thought of as a generalized extreme value theory. Instead of looking at events at fixed, regular time-points, this theory assumes random waiting times between the observations. This theory provides a model for bursty events, where the waiting times between the observations are heavy tailed.\\

% This is where the abstract is placed. It should include a statement about the problem being addressed in the presentation (and paper, if submitted). Continue with a discussion of why it is important to address this problem. This may be followed by some summary information about the models and methods developed and/or used to address the problem. Conclude with a description of the key results and contributions that will be covered in the presentation (and paper).
\end{abstract}

{\bf Keywords}: CTRM; exceedances; extreme value statistics; bursts.


\setlength{\parindent}{0pt}

\section{Introduction}
Extreme Value theory deals with the observation of extreme events which are the maxima of a sequence of observations and admits that these events occur at regular intervals in time. Recently a new theory called Continuous Time Random Maxima gets much regard, which can be thought of as a generalized extreme value theory. Instead of looking at events at fixed, regular time-points, this theory assumes random waiting times between the observations.


\section{The Max-Renewal Process}

It is an well-known result from classical extreme value theory that $M(n)$ converges weakly to a generalized Extreme Value Distribution with shape parameter $\xi \in \mathbb R$, i.e.
\begin{align}
[M(n) - d(n)] / a(n) \stackrel{d}{\to} A,
\end{align}
where the distribution of $A$ is given by
\begin{align}
\PP(A \le z) = G(z) = \exp\left(-[1+\xi z]^{-1/\xi}\right), \label{Mises}
\end{align}
which \eqref{Mises} is the famous van-Mises representation for Extreme Value distributions. For $\xi>0$ one gets a Fréchet distribution, for $\xi<0$ a Weibull and for $(1+\xi z)=e$ a Gumbel distribution. We write ${\rm GEV}(\xi, \mu, \sigma)$ for the probability 
distribution of the random variable $\sigma A + \mu$. \\
The extremal limit theorem allows for an extension to a functional
limit: assume that the norming sequences $a(n)$ and $d(n)$ are as 
in Theorem~\ref{th:GP}. Then 
\begin{align*}
[M(\lfloor ct \rfloor) - d(c)] / a(c) \stackrel{d}{\to} A(t),
\quad c \to \infty.
\end{align*}
Convergence is in Skorokhod's $J_1$ topology, and the limit
process $A(t)$ is an \emph{extremal process}, with
finite-dimensional distributions given by
\begin{align*}
\PP(A(t_i)\leq x_i,1\leq i \leq d) 
= F_A(\wedge_{i=1}^d x_i)^{t_1} 
F_A(\wedge_{i=2}^d x_i)^{t_2-t_1}
\ldots F_A(x_d)^{t_d-t_{d-1}},
\end{align*}
with $F_A(x)$ being a GEV distribution. \\
In this text, we assume that the waiting times $W_k$ have a tail
parameter $\beta \in (0,1)$, i.e. 
$\PP(W_1 > t) \sim L(t) t^{-\beta}$ as $t \uparrow \infty$ for some slowly varying function $L(t)$.  
(We write $f(t) \sim g(t)$ if their quotient converges to $1$.)
The law of the $W_k$ then lies in the 
domain of attraction of a stable law, in the sense that the limit
\begin{align}\label{eq:sclt}
(W_1 + \ldots + W_n)/b(n) \overset{d}{\longrightarrow} D, 
\quad n \to \infty
\end{align}
exists, for a regularly varying scaling function 
$b(n) = n^{1/\beta} L(n) \in {\rm RV}_\infty(1/\beta)$, 
where $L(n)$ is slowly varying. 
The limit is then a positively skewed stable distribution, 
whose scale parameter is $1$ is $b(n)$ is chosen accordingly; 
that is, 
$\E[\exp(-sD)] = \exp(-s^\beta)$.
Moreover, the following functional limit theorem holds:
\begin{align}
(W_1 + \ldots + W_{\Floor{ct}})/b(c) \overset{d}{\longrightarrow} D(t), 
\quad c \to \infty
\end{align}
with convergence in the Skorokhod $J_1$ topology.
The limit $D(t)$ is a stable subordinator, i.e.\ an increasing
L\'evy process with Laplace transform $\exp(-t s^\beta)$.\\ 
It is well known (see e.g.\ \cite{limitCTRW}) that the renewal
process then satisfies the functional limit
\begin{align}
N(ct)/\tilde b(c) \cd E(t) = \inf\{r: D(r) > t\}, 
\quad c \to \infty
\end{align}
for a scaling function $\tilde b(c)$ which is 
asymptotically inverse to $b(c)$, in the sense
of \cite[p.20]{seneta}: 
\begin{align}\label{eq:tildeb}
b(\tilde b(c)) \sim c \sim \tilde b(b(c)).
\end{align}
Note that $\tilde b \in {\rm RV}_\infty(\beta)$ 
\cite{limitCTRW}.
The limit process $E(t)$ is called the \emph{inverse} stable
subordinator \cite{invSubord}.
Was hier noch alles rein muss:
\begin{itemize} 
\item extreme value distribution
\item convergence of M(n)
\item Überleiten zu CTRM, via heavy-tailed waiting times, etc.
\item $J_1$ convergence definition
\item $J_1$ convergence of M(nt)
\item defintion of extremal processes 
\end{itemize}
\begin{definition}
Let $(W,J),(W_1,J_1),(W_2,J_2),...$ be i.i.d. random vectors with $J_i \in [x_L, x_R]$ where $x_L, x_R \in [-\infty, \infty]$ which are the event magnitudes and $W_i > 0$, which represent the random waitings times between the observations of the event magnitudes. Define the renewal process as
\begin{align} \label{eq:renewal-process}
N(t) = \max\{n \in \mathbb N: S(n) \le t\},
\end{align}
and let $M(n) := \bigvee_{i=1}^n J_i$ denote the maximum. Then the process
\begin{align}
V(t)=M(N(t)) = \bigvee_{k=1}^{N(t)} J_k, \quad t \ge 0.
\end{align}
is called a CTRM (Continuous Time Random Maxima) process and furthermore
\begin{align}
\tilde V(t) = \bigvee_{k=1}^{N(t)+1} J_k, \quad t \ge 0.
\end{align}
a OCTRM (Oracle Continuous Time Random Maxima).
\end{definition}
A CTRM can be regarded as a maximum process $M(n)$, time-changed by the
renewal process $n = N(t)$. 

\begin{proposition}
The CTRM and OCTRM processes are Semi-Markov processes: 
Let $\mathcal R = \{0\} \cup \{\sum_{k=1}^n W_k, n \in \mathbb N\}$ denote the set of renewal times, and let 
\begin{align*}
Z(t) = t - \sup((-\infty, t] \cap \mathcal R)
\end{align*}
be the \emph{age}, i.e.\ the time passed since the last renewal. Then the bivariate processes 
\begin{align*}
(V(t), Z(t)), \quad (\tilde V(t), Z(t)), \quad t \ge 0,
\end{align*}
are Markov processes with respect to their natural filtrations $\mathcal F_t = \sigma(V(s): s \le t)$ and $\tilde{\mathcal F}_t = \sigma(\tilde V(s): s \le t)$. 
\end{proposition}

The proof is deferred to Section \ref{sec:records}.

In XX it was shown that the CTRM $V(t) = M(N(t)$ and the OCTRM $\tilde V(t)$ have the following functional scaling limit in the Skorokhod $J_1$ topology: 
\begin{align*}
[V(ct) - d(\tilde b(c))] / a(\tilde b(c)) \stackrel{d}{\to} 
A(E(t)), \quad c \to \infty.
\end{align*}
Hence its scaling limit results from the time-change .....
We set $M(0) = x_L$, and $\max\{\} = 0$. 



\section{Scaling limits of exceedance and exceedance time}

\begin{definition}
The \textbf{exceedance time} and the \textbf{exceedance} of level $\ell \in [x_L,x_R]$ are
defined by
\begin{align*}
T_\ell &= \inf\{t: V(t) > \ell\}, 
&
X_\ell &= V(T_\ell) - \ell .
\end{align*}
\end{definition}

In the following proposition we list some results about the relationship between the exceedances and their exceedance times.

\begin{proposition}\label{lem:independence}
Let $(V_t)_{t \geq 0}$ resp. $(\tilde{V}_t)_{t \geq 0}$ be a (coupled) CTRM resp. OCTRM process, and let $\ell \in [x_L,x_R]$.
\begin{enumerate}
\item[(i)]
The exceedances $(X_{\ell, k})_{k \in \mathbb N}$ and exceedance times 
$(T_{\ell, k})_{k \in \mathbb N}$ are both i.i.d.
\item[(ii)]
In the OCTRM case, the exceedance $X_{\ell, k}$ and the corresponding exceedance time $T_{\ell, k}$ are independent for each $k \in \mathbb N$. 
\item[(iii)]
In the CTRM case, the exceedance and exceedance time are not necessarily independent. 
\end{enumerate}
\end{proposition}\begin{proof}
To write down!!
\end{proof}
The main purpose of this section is, to get the asymptotic distribution of the exceedance and the exceedance time as well as scaling limit theorems for their corresponding processes. It is well-known, that the excess function of $J$ is asymptotically Generalised Pareto distributed, i.e.
\begin{align*}
\PP(J_1 - \ell > y | J_1 > \ell) 
\approx (1+ \xi y / \tilde \sigma)^{-1/\xi} \rightarrow 1 - H(y) \text{ as } \ell \uparrow x_R,
\end{align*}
where $\tilde \sigma :=\sigma + \xi(\ell-\mu)$, $y>0$ and $1+\xi y/\tilde \sigma >0$. A distribution with CDF $H(y)$ is said to be from the \textbf{Generalised Pareto family}
$GP(\xi,\bar \sigma)$.\\  

\begin{theorem}(Verteilungskonvergenz der exceedance time)
For fixed $l>0$
\begin{align}
b(c) T(cl) \Rightarrow H(l) \text{ as } c \rightarrow \infty
\end{align} 
Where the CDF of $H(l)$ is given by
\begin{align*}
XX
\end{align*}
\end{theorem}

We can even show that the exceedance time process $(T_l)_{l \in \N}$ in converges to the inverse of CTRM resp. OCTRM limit in the $M_1$-topology.


\begin{theorem}(M1 convergence of the exceedance)
We have 
\begin{align*}
(T(cl))_{l \in [x_L,x_R]}  \rightarrow (D(A^{-1}(l)))_{l \in [x_L,x_R]} 
\end{align*}
\end{theorem}
Since the inverse function on XX is not continuous in Skorokhod's $J_1$ topology, we only get the convergence in the $M_1$ topology.\\
In XX and before in XX without noting it, it was shown, that in the uncoupled case the exceedance time is asymptomatically Mittag-Leffler distributed. This surprisingly holds in the coupled case as well. For$\beta \in (0,1)$, a standard Mittag-Leffler 
random variable $Y$ is positive with Laplace transform
$\E[\exp(-s Y)] = 1/(1+s^\beta)$. For $\sigma > 0$, we then write 
${\rm ML}(\beta, \sigma)$ for the distribution of $\sigma Y$. \\


\begin{theorem}(Asymptotic distribution of the exceedance (magnitudes))
We have 
\begin{align*}
(X(cl))_{l \in (0,1)} \rightarrow (A(A^{-1}(l)))_{l \in (0,1)} - l
\end{align*}
\end{theorem}

To proof this, we will first proof the following lemma.

\begin{lemma} 
Assume $(x_n)_{n \in \N} \subset \D$ and we have
\begin{align*}
x_n \rightarrow x  \text{ and }  x_n^{-1} \rightarrow x^{-1}. 
\end{align*}
Then it follows, that
\begin{align*}
x_n \circ x_n^{-1} \rightarrow x \circ x^{-1}.
\end{align*}
\end{lemma}


\section{Joint distribution of exceedance and exceedance time}


\section{Point process characterization of records}
\label{sec:records}

%It is self-similar with exponent $\beta$
%\cite{limitCTRW}, non-decreasing, and the (regenerative, random) set 
%$\mathcal R$ of its points of increase is a fractal with dimension $\beta$ 
%\cite{Bertoin04}.
%$E(t)$ is hence a model for time series with intermittent, `bursty'
%behaviour, for the following reasons:
%\begin{itemize}
%\item
%If $t \in \mathcal R$, for any $\epsilon > 0$ the interval
%$(t, t+ \epsilon)$ almost surely contains uncountably many other points of $\mathcal R$ (a ``bursty'' period)
%\item 
%If $t \notin \mathcal R$, then there is $\epsilon > 0$ such that
%$(t, t+ \epsilon)$ contains no other 
%\item [ii)]
%$\mathcal R$ has Lebesgue measure $0$, and hence $E(t)$ is
%constant at any ``randomly'' chosen time $t$. 
%\end{itemize}
%Having only two parameters ($\alpha \in (0,1)$ and a scale parameter)
%the inverse stable subordinator hence models scaling limits of heavy-tailed 
%waiting times parsimoniously.

The book \cite[Chapter 4]{resnick2013extreme} nicely characterizes the structure of records of extreme value processes. Record values and inter-record times follow a bivariate Poisson Point Process, whose intensity measure depends on the underlying distribution of event magnitudes. We generalize these results for max-renewal processes with regularly varying waiting times with infinite mean. 

A \textit{record time} $L$ for a CTRM $V(t)$ is defined by 
\begin{align*}
V(L) > V(t), \quad t < L.
\end{align*}
Paths of CTRMs are piecewise constant, non-decreasing and right-continuous, hence the above is well-defined and the $L$'s form an increasing sequence $\{L_n\}$. The \textit{record values} are then defined as $V(L_n)$. Clearly, the $L_n$ lie in the set $\mathcal R$ of renewal times, i.e.\ $L_n = W_1 + \ldots + W_{\tau(n)}$ for some integer $\tau(n)$. 

Given a record at $L_n$, the next record time $L_{n+1}$ occurs with the first event magnitude $J_k, k \ge \tau(n)$, which is larger than $V(L_n)$. As the $J_k$ are i.i.d., $\tau(n+1) - \tau(n)$ must be geometrically distributed, with parameter $p = 1 - F_J(V(L_n))$, and hence 
\begin{align} \label{eq:geometric-sum}
L_{n+1} - L_n \stackrel{d}{=} \sum_{k=1}^{N(p)} W_k
\end{align}
where $N(p) \sim {\rm Geom}(p)$, independent of the $W_k$. 

The record values $V(L_n)$ form a Markov chain (see e.g.\ \cite{resnick2013extreme}), and moreover we may characterize the record structure of CTRMs as follows: 

\begin{proposition}
\begin{enumerate}
\item
$\{V(L_n), n \ge 1\}$ is a Markov chain with stationary transition probabilities satisfying
\begin{align*}
\Pi(x, (y,\infty)) = 
\begin{cases}
(1-F(y)) / (1-F(x)), & y > x
\\
1, & y \le x
\end{cases}
\end{align*}
\item
$\{V(L_n), n \ge 1\}$ are the points of a Poisson random measure on $(x_l, x_0)$ with intensity measure
\begin{align*}
R(a,b] = R(b) - R(a)
\end{align*}
where $R(t) = -\log(1-F(t))$. 
\item
If $F_J$ is continuous, $\{V(L_n), L_{n+1} - L_n, n \ge 1\}$ are the points of a bivariate Poisson random measure on
$(x_l, x_0) \times (0,\infty)$ with intensity measure
\begin{align*}
\mu^*(dx, dy) = R(dx) F_W^{*N(1-F_J(x))}(dy),
\end{align*}
where $F_W^{*N(1-F_J(x))}(dy)$ is the convolution of the measure $F_W(dy)$ with itself, taken an independent geometric number of times, with success probability $p = 1-F_J(x)$.
\end{enumerate}
\end{proposition}

\begin{proof}
Parts 1 \& 2 are proven in 
\cite[Proposition 4.1 (i) \& (iii)]{resnick2013extreme}.
Part 3 follows from \cite[Proposition 3.8]{resnick2013extreme}, 
applied to the transition function $K(x,dy)$ corresponding to the law \eqref{eq:geometric-sum} of $L_{n+1} - L_n = y$, conditional on $V(L_n) = x$. 
\end{proof}









%\section{Conclusion}
\section{Conclusion}


{\bf Acknowledgements.} P. Straka was supported by the Australian Research Council’s Discovery Early Career Research Award DE160101147.


%\bibliographystyle{alpha}
%\bibliography{CTRMstats}

\begin{thebibliography}{KKBK12}

\bibitem[Bar05]{Barabasi2005}
Albert~L{\'{a}}szl{\'{o}} Barab{\'{a}}si.
\newblock {The origin of bursts and heavy tails in human dynamics}.
\newblock {\em Nature}, 435(May):207--211, 2005.

\bibitem[BGST06]{beirlant06Book}
Jan Beirlant, Yuri Goegebeur, Johan Segers, and Jozef Teugels.
\newblock {\em {Statistics of extremes: theory and applications}}.
\newblock John Wiley \& Sons, 2006.

\bibitem[B{\v{S}}14]{Basrak2014}
Bojan Basrak and Drago {\v{S}}poljari{\'{c}}.
\newblock {Extremal behaviour of random variables observed in renewal times}.
\newblock jun 2014.

\bibitem[Cah13]{Cahoy2013}
Dexter~O. Cahoy.
\newblock {Estimation of Mittag-Leffler Parameters}.
\newblock {\em Commun. Stat. - Simul. Comput.}, 42(2):303--315, feb 2013.

\bibitem[Haw71]{hawkes1971point}
Alan~G Hawkes.
\newblock {Point spectra of some mutually exciting point processes}.
\newblock {\em J. R. Stat. Soc. Ser. B}, pages 438--443, 1971.

\bibitem[HS16]{Heesa}
Katharina Hees and Hans-Peter Scheffler.
\newblock {On joint sum/max stability and sum/max domains of attraction}.
\newblock pages 1--31, jun 2016.

\bibitem[HS17]{Hees16}
Katharina Hees and Hans-Peter Scheffler.
\newblock {Coupled Continuous Time Random Maxima}.
\newblock pages 1--24, feb 2017.

\bibitem[HTSL10]{HLS2010b}
B.I. Henry, {T.A.M. Langlands}, and Peter Straka.
\newblock {An introduction to fractional diffusion}.
{\em Complex Phys. Biophys.
  Econophysical Syst.} World Scientific, 2010.

\bibitem[KKBK12]{Karsai2012}
M{\'{a}}rton Karsai, Kimmo Kaski, Albert~L{\'{a}}szl{\'{o}} Barab{\'{a}}si, and
  J{\'{a}}nos Kert{\'{e}}sz.
\newblock {Universal features of correlated bursty behaviour}.
\newblock {\em Sci. Rep.}, 2, 2012.

\bibitem[MK00]{Metzler2000}
Ralf Metzler and Joseph Klafter.
\newblock {The random walk's guide to anomalous diffusion: a fractional
  dynamics approach}.
\newblock {\em Phys. Rep.}, 339(1):1--77, dec 2000.

\bibitem[MS04]{limitCTRW}
Mark~M Meerschaert and Hans-Peter Scheffler.
\newblock {Limit Theorems for Continuous-Time Random Walks with Infinite Mean
  Waiting Times}.
\newblock {\em J. Appl. Probab.}, 41(3):623--638, sep 2004.

\bibitem[MS08]{MeerschaertStoev08}
Mark~M Meerschaert and Stilian~A Stoev.
\newblock {Extremal limit theorems for observations separated by random power
  law waiting times}.
\newblock {\em J. Stat. Plan. Inference}, 139(7):2175--2188, jul 2008.

\bibitem[MS13]{invSubord}
Mark~M Meerschaert and Peter Straka.
\newblock {Inverse Stable Subordinators}.
\newblock {\em Math. Model. Nat. Phenom.}, 8(2):1--16, apr 2013.

\bibitem[Sen76]{seneta}
E~Seneta.
\newblock {\em {Regularly Varying Functions}}, volume 508 of {\em Lecture Notes
  in Mathematics}.
\newblock Springer-Verlag, Berlin, 1976.

\bibitem[SS83]{Sumita1983}
J.~George Shanthikumar and Ushio Sumita.
\newblock {General shock models associated with correlated renewal sequences}.
\newblock {\em J. Appl. Probab.}, 20(3):600--614, 1983.

\bibitem[ST04]{ST04}
Dmitrii~S Silvestrov and Jozef~L. Teugels.
\newblock {Limit theorems for mixed max-sum processes with renewal stopping}.
\newblock {\em Ann. Appl. Probab.}, 14(4):1838--1868, nov 2004.

\bibitem[VTK13]{Vajna2013}
Szabolcs Vajna, B{\'{a}}lint T{\'{o}}th, and J{\'{a}}nos Kert{\'{e}}sz.
\newblock {Modelling bursty time series}.
\newblock {\em New J. Phys.}, 15(10):103023, oct 2013.

\end{thebibliography}


\end{document}
